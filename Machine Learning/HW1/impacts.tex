
\section{Broader Impacts}
\label{sec:impacts}

What this model offers the wine industry is a mathematical understanding of the relationships between wine features and rating. People can describe the qualitative features they experience when tasting a wine, but implementing a model such as this provides an opportunity to carry out this type of testing at a faster rate. Additionally, the model that is generated can provide valuable insights into how humans’ rate wine. Individuals could identify through our model which traits affect the rating the greatest. Wineries can profit off this information by adjusting their recipes using this data generated by regressing these data sets. \\\\
Despite the automation that machine learning algorithms provide, the implementation and design of these methods depend upon human decision making. Implicit bias that the developer or data source may possess can cause an algorithm to behave improperly, resulting in inaccurate predictions \cite{mehrabi2019survey}. By the same token, these biases can worsen inequalities if a machine learning model is rooted in logic that negatively scores specific racial, ethnic, socioeconomic, or other personal traits. Preventing this and other means of biased discrimination must be considered when implementing any machine learning algorithm that has real-world applications \cite{inproceedings}. \\\\
For our wine-prediction algorithm, the uses of this tool are specific enough that we are confident there would be minimal unjust consequences that result from our implementation. Because the quality of wine was rated by humans, there is an additional opportunity for personal bias to affect our model’s predictions. One possible inequity could result from our algorithm favoring wines from specific brands or companies. If we were to implement this algorithm in a context that influenced global wine consumption, we would have to perform additional testing to see whether this algorithm might harm the wine market by reducing competition between wineries. 
