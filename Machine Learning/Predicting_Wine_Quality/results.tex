
\section{Results}
\label{sec:results}

We fit our two models for red wine and white wine using their respective optimal hyperparameters and found distinct differences between the two types of wine. As for the red wine, the training dataset had a coefficient of determination of 36.56, and the test dataset had a coefficient of determination of 35.06. For the white wine, the training dataset had a coefficient of determination of 27.87, and the test dataset had a coefficient of determination of 28.51. This is tabulated in figure~\ref{tab:example}. In addition, as seen in figure~\ref{fig:hptuning} above, the white wine dataset had a lower mean squared error than the red wine dataset. \\\\

\begin{figure}[htb]
  \centering
  \begin{tabular}{|c|c|c|} 
    \hline
     & Red Wine & White Wine \\
    \hline
    Training Set & $36.56$ & $27.87$ \\
    Test Set & $35.06$ & $28.51$\\
    \hline
  \end{tabular}

  % As with figures, *every* table should have a descriptive caption
  % and a label for ease of reference.
  \caption{Coefficient of determination ($R^2$) values for both the training and test sets for red and white wine.}
  \label{tab:example}

\end{figure}

This indicates that our model for red wines fit the data slightly better than the model for white wines. One of the biggest differences between the red and white wine datasets (other than the inherent physicochemical differences in the wine) was the size of the dataset. The white wine dataset was roughly three times larger than the red wine dataset, which meant that it trained on far more data. Perhaps we did not address overfitting as dutifully, which may explain the lower mean squared error and coefficient of determination. In terms of the choice of regression model, the minor difference between lasso and ridge regression may indicate the need for further research and experimentation with these models and data. \\\\
In addition, the optimized feature coefficients differed between red and white wine data. Values for each coefficient for both datasets is tabulated in figure~\ref{tab:coeff}. A lot of the features were weighted similarly between the two types of wine such as residual sugar, free sulfur dioxide, total sulfur dioxide, and density. However, other features like chlorides are drastically different impacts on their respective regression models, indicating chlorides in red wine may be more influential to the overall wine quality.


\begin{figure*}[htb]

  \begin{tabular}{|c|c|c|c|c|c|c|} 
    \hline
     & Fixed Acidity & Volatile Acidity & Citric Acid & Residual Sugar & Chlorides & Free Sulfur Dioxide\\
    \hline
    Red Wine & $-.016$ & $-0.890$ & $0.227$ & $0.024$ & $-2.176$ & $0$\\
    White Wine & $-0.023$ & $-1.319$ &  $0$ & $0.023$ & $0$ & $0.009$ \\
    \hline
  \end{tabular}
 \\\\\\\
\begin{tabular}{|c|c|c|c|c|c|} 
    \hline
     & Total Sulfur Dioxide & Density & pH & Sulphates & Alcohol\\
    \hline
    Red Wine & $-0.003$ & $0$ & $-0.450$ & $0.926$ & $0.261$\\
    White Wine & $-0.003$ & $0$ & $0$ & $0.496$ & $0.357$ \\
    \hline
 \end{tabular}

  % As with figures, *every* table should have a descriptive caption
  % and a label for ease of reference.
  \caption{Features and their given weights assigned by the regression model for both red and white wine.}
  \label{tab:coeff}

\end{figure*}

%    Red Wine & $-.016$ & $-0.890$ & $0.227$ & $0.024$ & $-2.176$ & $0$ & $-0.003$ & $0$ & $-0.450$ & $0.926$  $0.261$\\
%White Wine & $-0.023$ & $-1.319$ &  $0$ & $0.023$ & $0$ & $0.009$ & $-0.003$ & $0$ & $0$ & $0.496$ & $0.357$ \\