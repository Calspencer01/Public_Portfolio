
% The \section{} command formats and sets the title of this
% section. We'll deal with labels later.
\section{Introduction}
\label{sec:intro}

Wine tasting has been a tradition of wine enthusiasts for centuries. Each taster evaluates the quality of the wine based on their own unique palate and generates consumer preferences. The process of rating wines is thought to be completely subjective, but what if one could predict how a particular wine is rated based on physicochemical characteristics of the wine like alcohol content and acidity. Employing this technique would revolutionize the wine consumer market through the creation of new wines and targeted advertising of certain wines. \\\\
An idea like that of predicting wine quality based on its characteristics becomes feasible when taking a regression modeling approach. Regression analysis will consider both the characteristics and ratings of pre-existing wines and use patterns in that data to rate wines that the model has not seen yet. Through this type of modeling, we can ideally rate any kind of wine both in existence and in imagination. In this paper, we followed this regression approach to see if we could accurately predict a subjective wine rating from objective inherent characteristics and previous ratings. \\\\
The data used to fuel our regression model came from the UCI Machine Learning Repository, created by Paulo Cortez, Antonio Cerdeira, Fernando Almeida, Telmo Matos, and Jose Reis. Data was split into two sets, one set contained data on the Portuguese "Vinho Verde" red wine variant and the other on the corresponding white wine variant. The following are the eleven physicochemical variables that were recorded for each wine: fixed acidity, volatile acidity, citric acid, residual sugar, chlorides, free sulfur dioxide, total sulfur dioxide, density, pH, sulphates, and alcohol. In addition, wine experts graded each wine on a scale of 0 to 10 with 0 representing “very bad” and 10 representing “very excellent.” The median of three experts’ grades was compiled as this variable. The red wine dataset contains 1,599 wines, and the white wine dataset contains 4,898 wines. \\\\
In this paper, we will detail the process of exploring the data, augmenting the data, generating and optimizing regression models for both red and white wines, and comparing the models’ effectiveness.
\\

% Citations: As you can see above, you create a citation by using the
% \cite{} command. Inside the braces, you provide a "key" that is
% uniue to the paper/book/resource you are citing. How do you
% associate a key with a specific paper? You do so in a separate bib
% file --- for this document, the bib file is called
% project1.bib. Open that file to continue reading...

% Note that merely hitting the "return" key will not start a new line
% in LaTeX. To break a line, you need to end it with \\. To begin a 
% new paragraph, end a line with \\, leave a blank
% line, and then start the next line (like in this example).

